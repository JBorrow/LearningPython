\section{Functions}

Functions are our way of cutting down on code re-use. Let's say you want to do
something several times in your code, for example add two to a number. You could
just have 'x = x + 2', but that would be inefficient - what happens when you
want to change it to being 'x = x + 3'? We can deal with this by using a
function.

\subsection{Function Structure}

Functions have a name, for example 'type'. They are then followed by brackets -
these must be included in \emph{all} functions, even if you don't need to pass
any arguments through them. An argument is something you 'give' to the function,
and are in the brackets:

\begin{lstlisting}
    >>> type(10)
    <type 'int'>
\end{lstlisting}

Here, we give the number '10' (an argument) to the function called 'type', which
we do by enclosing it in the brackets.

We can even have functions that take more than one argument:

\begin{lstlisting}
    >>> range(1, 10, 2)
    [1, 3, 5, 7, 9]
\end{lstlisting}

This built in function, called 'range()', takes one to three arguments. You can specify only one, and it will give you a list of that many numbers, starting at zero. For example:

\begin{lstlisting}
    >>> range(9)
    [0, 1, 2, 3, 4, 5, 6, 7, 8]
\end{lstlisting}

However, when we give it three arguments, we specify start, stop and step. So
the range function will give us a list of numbers starting at the first one,
stopping at the second one, with a step in between of the last number. This is a
very useful funciton - you will need to create many lists of numbers in your
time programming and this is much faster than writing them all out yourself!

\subsection{Defining Your own Functions}


