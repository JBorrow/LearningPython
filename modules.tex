\section{Loading External Modules}

Python can be used as a modular language, meaning that you keep different
functions in different files. This allows you to use a function more than once
without copying and pasting it into the new file.

Importing modules and calling functions from them is very simple. We will begin
by using a module written by someone else, and then we will move to looking at ones that we have written ourselves.

\subsection{Loading in a Module}

There is a module called `Numpy' that provides a lot of useful mathematical
tools for python users. If we want to load the module in, we just need to add
`import numpy' at the top of our file. If we then want to use functions from it,
for example the `array()' function, we write `numpy.array()'. For example:

\begin{lstlisting}
    >>> import numpy
    >>> ourlist = [1,2,3,4]
    >>> ourarray = numpy.array(ourlist)
    >>> type(ourarray)
    <type 'numpy.ndarray'>
\end{lstlisting}

This code snippet creates a list and then turns it into an array. We will learn
much more about the numpy module later, in it's own section.

\subsection{Writing and Using Your Own Modules}

It is very simple to write your own modules. We will begin by learning to use
files to store your python code.

\subsubsection{Writing a python File}

We have been running all of our programs interactively so far. Here, we are
going to use files to run our python scripts from - which is much easier if you
make a mistake (you can go back and edit it instead of having to write it all
again). It's similar to moving from using a pen and paper to using a word
processor.

If you haven't already chosen and downloaded an editor and become familiar with
using the terminal, please go and read those sections at the beginning of the
document as they will be used heavily here.

Open your text editor and write some python code! I would suggest performing a
simple `hello world' as your first program from file.

Write this:
\begin{lstlisting}
    hi = 'Hello World'
    print hi
\end{lstlisting}

Save your file, with a `.py' extension, for example `helloworld.py'.

Now, open your command prompt/terminal and change to the directory that the
python code is stored in (cd), and type:

\begin{lstlisting}
    python <yourfilename>.py
\end{lstlisting}

You should now have `Hello World' printed in your console!


