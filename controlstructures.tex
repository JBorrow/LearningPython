\section{Control Structures}

Control structures are how we get things done in programming, it is how the
computer makes decisions. We can ask the computer if something is true, for
example $7 > 4$, and then act on this information. First, we must introduce
booleans.

\subsection {Booleans}

Booleans are another type of variable, that are best introduced alongside these
structures. There are two possible states for a boolean variable, `True' or
`False'. In python it is very important that the first letter is capitalised.

\subsection{if Statements}

These are the foundation of control structures. What we say to the computer is,
if this condition is true, execute this bit of code. If it's not, then do this
instead. This turns out to be a very powerful tool. Here's an example:

\begin{lstlisting}
    >>> x = 3
    >>> if x < 4:
    ...     #if x is less than 4 execute this
    ...     print `x is less than 4'
    ... else:
    ...     #otherwise execute this
    ...     print `x is bigger than or equal to 4'
    ...
    x is less than 4
\end{lstlisting}

There are a few new things that we have introduced here, aside from the if
statement. We can add a comment in python by having a \# anywhere on the line.
If the interpreter finds a \#, it ignores all other characters after that.

This allows us to create comments that aid people inheriting our code figure
out what it means a lot more easily than having to pick through it line by line.
Commenting your code is immensely important and is a good habit to get into.
However, don't excessively comment your code as this will simply frustrate
whoever has to read it. As you do more and more programming you will learn what
the correct amount of comments is - this is not something that is easily taught!

\subsubsection{elif Statements}

If you have more than one condition you would like to check, you can use elif
statements. This is short for `else if' and is executed if the first if
statement is not true, but it is. For example:

\begin{lstlisting}
    >>> x = 5
    >>> if x < 4:
    ...     #if x is less than 4 execute this
    ...     print `x is less than 4'
    ... elif x < 10:
    ...     print `x is less than 10, but bigger than 4'
    ... else:
    ...     #otherwise execute this
    ...     print ``x is bigger than or equal to 4"
    ...
    x is less than 10, but bigger than 4
\end{lstlisting}

You can have as many elif statements as you would like, so you can check an
unlimited number of statements.


