\section{Introduction}

This little book should be your guide during this course. We will begin with
our first program - a simple 'hello world'. This program will aim to teach you
how to output text to the terminal. Hopefully, you have already read the
previous section on setting up python on your system, and now when you type
'python' in the terminal you are presented with three greater than signs, like
this:

\begin{minted}
    >>>
\end{minted}

\subsection{Strings and Variables}

Now, the first thing that we want to be able to do in a programming language is
declare variables (well, that's up for debate, but for me it is). We declare a
variable in python by using the '=' character. This means something different
than when it is used in mathematics - it is an \emph{assignment} operator rather
than an \emph{equality}.

Because python is a very 'loosely typed' language, this means we can set our
variables to any type - be it an integer, a floating point number (decimal) or
a 'string'. A 'string' is a set of characters, such as 'Hello World'. The
characters are enclosed in a set of quotes or inverted commas.

So if I want to give a variable called 'hi' a value of "Hello World!":

\begin{minted}
    >>> hi = "Hello World!"
\end{minted}

The next thing we want to be able to do is print some output to the screen. In
python, we do this by using 'print'. In python 2.x, this is a pseudofunction,
meaning it doesn't need brackets - we will learn more about this later. To print
the value of dog to the screen, we simply need to write:

\begin{minted}
    >>> print hi
    \bf{Hello World!}
\end{minted}

In this book, I will have statements that we type into the prompt/editor as
a regular monospaced font, and return values in bold.

\subsection{Types of Variable and Definitions}

Now we've got you hooked after writing your first computer program, we have to
cover some boring stuff. If you want to, you can skip this section and come back
if some of the words I start to use in the later sections confuse you. We'll
start by talking about the three basic variable types.

\subsubsection{Integers}

Integers are exactly what you would expect them to be. They can have almost any
value you want to give them. However, the interesting thing here is division.

When you divide two integers in python, it returns an integer too. This is
accomplished by simply discarding the remainder. For example:

\begin{minted}
    >>> 5/2
    \bf{2}
    >>> 20/4
    \bf{5}
    >>> 121/120
    \bf{1}
\end{minted}

This process is called 'integer division', rather than decimal or 'float
division'.

Integer division can cause some problems, and was changed in python 3 (however
we're using python 2) so that the division automatically converts the values to
decimals.

To get the remainder from this division, we use the 'modulo' operator. This
operator is represented by the percentage sign, \%. For example:

\begin{minted}
    >>> 5\%2
    \bf{1}
    >>> 20\%4
    \bf{0}
    >>> 121/120
    \bf{1}
\end{minted}

\subsubsection{Floating-point Numbers}
