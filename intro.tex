\section{Introduction}

This little book should be your guide during this course. We will begin with
our first program - a simple `hello world'. This program will aim to teach you
how to output text to the terminal. Hopefully, you have already read the
previous section on setting up python on your system, and now when you type
`python' in the terminal you are presented with three greater than signs, like
this:

\begin{lstlisting}
    >>>
\end{lstlisting}

\subsection{Strings and Variables}

Now, the first thing that we want to be able to do in a programming language is
declare variables (well, that's up for debate, but for me it is). We declare a
variable in python by using the `=' character. This means something different
than when it is used in mathematics - it is an \emph{assignment} operator rather
than an \emph{equality}.

Because python is a very `loosely typed' language, this means we can set our
variables to any type - be it an integer, a floating point number (decimal) or
a `string'. A `string' is a set of characters, such as `Hello World'. The
characters are enclosed in a set of quotes or inverted commas.

So if I want to give a variable called `hi' a value of ``Hello World!":

\begin{lstlisting}
    >>> hi = `Hello World!'
\end{lstlisting}

The next thing we want to be able to do is print some output to the screen. In
python, we do this by using `print'. In python 2.x, this is a pseudofunction,
meaning it doesn't need brackets - we will learn more about this later. To print
the value of hi to the screen, we simply need to write:

\begin{lstlisting}
    >>> print hi
    `Hello World!'
\end{lstlisting}

\subsection{Types of Variable}

Now we've got you hooked after writing your first computer program, we have to
cover some boring stuff. If you want to, you can skip this section and come back
if some of the words I start to use in the later sections confuse you. We'll
start by talking about the four basic variable types.

\subsubsection{Integers}

Integers are exactly what you would expect them to be. They can have almost any
value you want to give them. However, the interesting thing here is division.

When you divide two integers in python, it returns an integer too. This is
accomplished by simply discarding the remainder. For example:

\begin{lstlisting}
    >>> 5/2
    2
    >>> 20/4
    5
    >>> 121/120
    1
\end{lstlisting}

This process is called `integer division', rather than decimal or `float
division'.

Integer division can cause some problems, and was changed in python 3 (however
we're using python 2) so that the division automatically converts the values to
decimals.

To get the remainder from this division, we use the `modulo' operator. This
operator is represented by the percentage sign, \%. For example:

\begin{lstlisting}
    >>> 5%2
    1
    >>> 20%4
    0
    >>> 121%120
    1
\end{lstlisting}

\subsubsection{Floating-point Numbers}

Floating point numbers are what mathematicians call `real' numbers. They can
have almost any decimal value, and their division works in the normal way.

You can force a variable to be `cast' as a float rather than as a integer by
simply putting a full stop after it, or by casting it through the function
`float()':

\begin{lstlisting}
    >>> a = 2
    >>> b = 3.
    >>> c = float(4)
    >>> d = 1035.2342345234
    >>> type(a)
    <type `int'>
    >>> type(b)
    <type `float'>
    >>> type(c)
    <type `float'>
    >>> type(d)
    <type `float'>
\end{lstlisting}

\subsubsection{Strings}

Strings, as explained above, are a `string' of characters. They can be used to
put `words' into your program, or more excitingly, we can use the program to
operate on strings. For example, we can go through each character in the string,
and move it up two in the alphabet to `encrypt' it!

Examples like these will be covered later in the course - so get excited!

\subsection{Lists}

Lists are just that - lists of variables. We declare a list in the following
way:

\begin{lstlisting}
    >>> list = [1,2,3,4,5,11]
\end{lstlisting}

In a list, each variable is separated by a comma. We can fill lists with any
type of object, be it a list, a float, even other lists!

\subsubsection{Indexing}

Lists are indexed to allow us to find values in them easily. The first value in
the list has an index of 0, the second 1, the third 2, and so on. Lists are also
negatively indexed, meaning we can find the last object in the list with an
index of -1, second last -2, and so on.

We access the values in the list by putting the index in square brackets
following the list. It's best illustrated using an example.

\begin{lstlisting}
    >>> list[0]
    1
    >>> list[3]
    4
    >>> list[-1]
    11
    >>> list[-3]
    4
\end{lstlisting}

Strings are also indexed. For example:

\begin{lstlisting}
    >>> hi = `Hello World'
    >>> hi[0]
    'H'
    >>> hi[-1]
    'd'
\end{lstlisting}

This allows us to do interesting things, for example we can check if the
extension of a file have the name of is `.png', etc.

\subsection{Definitions}

In this section we will define what some of the terms mean that are scattered
throughout the book.

\subsubsection{Iteratable}

An Iteratable object is one where you can go through and use each object. They
are objects that have indexes, so strings and lists are examples of iteratable
objects. They allow us to write loops using their values, which we will learn
about in more detail later.

\subsubsection{Function}

A function is a construct that is used to do something over and over again.
An example of a function is the `type()' function used above. This function
takes a variable and then returns its type.
