\section{Iterating}

If you have a large set of data, and you want to perform operations on all of
them, then you could do operations on each one by hand. However, this will take
a very long time, and what if someone wants you to do it again, with different
data?

This is where computers and programming comes in. We can deal with large volumes
of information with ease and perform huge amounts of calculations in a very
short time. The following statements will show you how to \emph{iterate} over
\emph{iteratable} statements.

\subsection{The while Loop}

The while loop keeps on going until some condition is met. This includes the
infamous 'while true' loop - which will keep on going forever until either the
program crashes or the user tells it to stop.

Here's an example of a while loop:

\begin{lstlisting}
    >>> x = 0
    >>> while(x < 3):
    ... print x
    ... x = x + 1
    ...
    0
    1
    2
\end{lstlisting}

This is interesting, because if we hadn't initialised the variable 'x' as 0 to
begin with, then no code inside the while loop would have been executed, as how
can you check if 'something' is less than 3 if that thing is not a number?

It also showcases what we mean when we say that the '=' sign is an assignment,
rather than an equality. We can - quite rightfully - say that 'x = x + 1',
whereas if this was mathematics then that would be an impossible statement. What
this really means is 'take x and set it equal to what it used to be, with one
added'. 

\subsection{The for Loop}

This is where iteratable structures come in (which you would know about if you
had read the 'boring' bit that I told you could skip - if you're confused then
now would be a good time to go back and read it). With the for loop, we can say
take this structure, and one at a time give me something from it and let me
do something to it. For example:

\begin{lstlisting}
    >>> l = ['a', 'b', '1', '2', '7']
    >>> for item in l:
    ...     print item
    ...
    a
    b
    1
    2
    7
\end{lstlisting}

We can even iterate through strings:

\begin{lstlisting}
>>> hi = "Hello World"
>>> for character in hi:
...     print character
...
H
e
l
l
o

W
o
r
l
d
\end{lstlisting}
